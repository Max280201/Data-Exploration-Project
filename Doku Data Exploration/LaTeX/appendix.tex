% !TEX root =  master.tex
\chapter{Beschreibung des Quellcodes}
\section{Ausführen des Programms}
\begin{enumerate}
	\item Klonen des Github-Repository unter dem Link https://github.com/Max280201/Data-Exploration-Project (Branch: main)
	\item Öffnen der CMD und navigieren in den Git-Hub-Ordner
	\item CMD-Command 'py -3.X -m venv Hoyzer' (Bash: 'python3.X -m venv HoyzerVenv') ausführen (erstellt die virtuell Environment, ersetzen von X durch die vorhandene Python-Version)
	\item \grqq HoyzerVenv/Scripts/activate.bat\grqq in CMD ausführen (aktiviert die virtuell Environment)
	\item CMD-Command \grqq pip install -r requirements.txt\grqq{} ausführen
	\item Auswählen der python version der venv als Kernel; Ausführen der Python-Datei/ Notebooks anschließend möglich
\end{enumerate}

\section{Erklärung der einzelnen Bestandteile}
\begin{itemize}
	\item preprocessing\_pipeline\_v1.ipynb: kombiniert die csv-Dateien der Saisonspiele und die Marktwerte; außerdem Datenvorverarbeitung und Berechnung weiterer Features
	\item modelltraining\_prod\_v4.py: Modelltraining eines NN (sowohl eines, um die letzte Saison vohrerzusagen, als auch eines, um zufällige Spiele des Bereichs vorherzusagen) und kurze Evaluation des Modells
	\item modelltraining\_test\_area\_v4.ipynb: enthält alle relevanten getesteten Techniken für das Modell und Alternativen (Hyperparametertuning, Ensemble-Learning, Poission-Prediction);
	
	Anmerkung: Random Search kann zu dem Fehler \grqq Access denied\grqq{} führen, dies lässt sich durch eine geringere Anzahl an Wiederholungen beheben (https://github.com/keras-team/keras-tuner/issues/339)
	
	\item translate\_betting\_odds.ipynb: Auswertung und Vorverarbeitung der Wettquoten
	\item evaluate\_model\_odd\_predictions\_v2.ipynb: Vergleich des trainierten Modells mit Wettanbietern und Auswertung
\end{itemize}