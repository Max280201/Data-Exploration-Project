% !TEX root =  master.tex
\chapter{Theoretische Grundlagen}

\section{Related Work}
\autocite[Vgl.][]{Tableau.03112021}

Bei der Recherche wurde mit der Suche nach einer geeigneten Datenbasis begonnen. Nach längerer Suche wurde sich für den Datensatz von football-data.co.uk entschieden \autocite{.03072022}. Beim Preprocessing wurden einige relevante Features hinzugefügt. Begonnen wurde hierbei mit Elo und Angriffs- und Verteidigungsstärke \autocite{Accso.14102021}. Als Vergleichsprojekt wurde ein mathematischer Ansatz gefunden. Dieser ist mit R programmiert und schafft laut angaben der Quelle 64\% accuracy \cite{Doan.15.3.2019}. Für die grundlegende Idee für den Business Use Case wurde eine weitere Quelle herangezogen \cite{Hartley.05102022}.

\section{Verwendete Technologien und Bibliotheken}
Für die Datenvorbereitung werden klassische Python-Module wie Pandas, Sklearn und Matplotlib genutzt.

Bei der Modellimplementierung wurde vorerst ein Decision Tree Classifier genutzt. Dieser stammt aus dem Sklearn-Modul tree. Nach Ausprobieren dieses Classifiers wurde sich für das Ensemble-Learning entschieden. Dabei nutzt man mehrere Algorithmen um sie zeitgleich zu vergleichen und potenziell zu vereinen. Die Algorithmen wie beispielsweise Logistic Regression oder Gaussian Naive Bayes stammen auch von Sklearn. Final wurde ein Neural Network implementiert. Dazu wurde Keras genutzt. Aus Keras wurde außerdem eine Random Parameter Search zur weiteren Optimierung angewandt. Durch Matplotlib wurden die Ergebnisse visualisiert. Außerdem wurde eine Poisson-Verteilung genutzt um einen mathematischen Ansatz zu implementieren.