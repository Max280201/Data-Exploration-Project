% !TEX root =  master.tex
\chapter{Praktischer Teil}

\section{Umsetzung}

Eine der wohl wichtigsten Entscheidungen bei der Implementation eines Machine Learning Modells sind die ausgewählten Rohdaten. Nach intensiver Recherche wurde sich für einen Datensatz entschieden welcher Informationen über alle Bundesliga-Spiele seit der Saison 2005/2006 enthält. Dieser Rohdatensatz besitzt 65+ Features. Im Preprocessing wurden einige unvollständige Spieldaten gelöscht und viele unnötige Features - wie die Betting Odds der Anbieter. Nach Harmonisierung der Daten wurden die Features auf die Anzahl 21 reduziert.  (Hier Grafik von Datensatz). Dadurch, dass die Zielvariablen im Datensatz bereits vorliegen handelt es sich um ein Supervised Learning-Problem. Dieses wurde wie bereits im Theorieteil erwähnt mit drei verschiedenen Methoden angegangen. Dem Ensemble Learning, einem Neural Network und einem mathematischen Ansatz.

\section{Ergebnisse}

Angefangen beim Ensemble Learning kann man verschiedene Accuracys je Algorithmus erkennen (Grafik zeigen). Hierbei sind die Logistic Regression und Linear Discriminant Analysis am Besten. Das Stacking Modell welches die anderen Modelle vereint schneidet auch gut ab. Alle drei liegen bei einer Accuracy von etwas unter 50 Prozent.

Bei Betrachtung des Neural Networks fällt auf, dass die Accuracy schon nach wenigen Epochen auf etwas unter 50 Prozent konvergiert (Grafik). Nach einer Random Parameter Search konnte diese auf 53 Prozent angehoben werden. Zuletzt haben wir mit der Poisson-Verteilung des mathematischen Ansatzes auch eine Accuracy von etwa 50 Prozent erreicht (Grafik). 
