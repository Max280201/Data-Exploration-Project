% !TEX root =  master.tex
\chapter{Zufall und Vorhersehbarkeit im Fußball}
	
Was macht Fußball so spannend und beliebt? Sepp Herberger hat dazu eine klare Meinung:
\glqq Fußball ist deshalb spannend, weil niemand weiß, wie das Spiel ausgeht.\grqq 

Dass Sepp Herberger teilweise Recht hat bestätigen immer wieder vermeintliche Underdogs, die den klaren Favoriten in einem Spiel schlagen oder sogar debütieren können. Hierfür gibt es Statistiken, die den Effekt erklären: So fallen 47\% aller Tore zufällig, also durch einen Fehler in der Verteidigung oder einen abgefälschten Schuss \autocite[Vgl.][]{.06102018}. Außerdem ist jeder dritte Ballkontakt ein unvorhersehbares Ereignis. Es wird also beinahe unmöglich einen perfekten Spielzug mit anschließendem Torerfolg zu planen. \autocite[Vgl.][]{.}

Dennoch gibt es auch Komponenten, die das Gegenteil vermuten lassen.
So wissen wir aus Berechnungen, dass die Mannschaft mit dem höheren Marktwert zu 48\% gewinnt, das Heimteam zu 45\% oder das Team mit der besseren Tabellenplatzierung zu 44\%. 
Inwieweit ist Fußball also etwa durch Machine Learning vorhersehbar oder überwiegt doch die Zufallskomponente? Und ist es tatsächlich möglich Anbieter wie Tipico und Bwin durch ein mathematisches Modell zu schlagen? Diese beiden Fragen versuchen wir im Rahmen der Arbeit zu beantworten.

Das erfolgreiche Vorhersagen von Fußballspielen wäre auch mit einem enormen wirtschaftlichen Nutzen versehen. So kann man durch eine solche Prediction potenziell starken Profit bei etwaigen Wettanbietern erzielen. Andererseits kann man diese Software auch an Wettanbieter verkaufen, damit diese ihre Quoten noch effizienter und genauer berechnen können.


