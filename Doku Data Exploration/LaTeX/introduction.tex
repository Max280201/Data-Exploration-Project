% !TEX root =  master.tex
\chapter{Einleitung}

\section{Hintergrund und Motivation}
Dieser Projektreport ist im Rahmen des Fachs Data Exploration entstanden. Das Ziel des Moduls ist die \glqq Anwendung von Methoden und Verfahren des maschinellen Lernens auf eine vorgegebene Datenbasis unter Laborbedingungen\grqq [Modulhandbuch]. Zusätzlich soll neben der informatischen Betrachtung auch der betriebswirtschaftliche Nutzen erörtert werden [vgl. Modulhandbuch].

Auf Basis dieser Vorgaben wurde das Thema des Projekts gesucht. Dabei ging es primär darum ein Themengebiet zu finden, welches sowohl breite Möglichkeiten für die informatische als auch die betriebswirtschaftliche Betrachtung bietet. Aufgrund der Interessen innerhalb der Gruppe wurde sich für das Thema \textbf{Bundesliga Match Predictions} entschieden. Wir wollten der Fragestellung auf den Grund gehen, ob es tatsächlich möglich ist, diese unzählig erscheinenden Faktoren des Fußballsspiels durch Data Science-Prozesse für eine Vorhersage nutzen zu können.

\section{Business Use Case}
Wie bereits erwähnt spielt die wirtschaftliche Betrachtung dieses Projekts neben der informatischen Arbeit eine primäre Rolle. Das Ziel einer Bundesliga Match Prediction liegt hier auf der Hand. Ist es tatsächlich möglich Anbieter wie Tipico und Bwin durch ein mathematisches Modell zu schlagen?

Durch eine solche Vorhersage kann man potenziell starken Profit bei etwaigen Wettanbietern erzielen. Andersrum kann man natürlich auch diese Software an Wettanbieter verkaufen, damit diese ihre Quoten noch effizienter und genauer berechnen können. 

Die Herangehensweise an dieses Projekt beginnt mit der richtigen Datenbasis. Durch diese kann man algorithmisch ein Modell erstellen, dass den genannten Business Use Case ermöglichen kann.