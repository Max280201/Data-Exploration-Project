% !TEX root =  master.tex
\chapter{Fazit}

Abschließend lässt sich feststellen, dass sich die Accuracy des Neural Networks
im Verlauf des Projekts Schritt für Schritt durch das Optimieren der Datenbasis
und des Modells erhöhen ließ. Ausschlaggebend hierfür sind vor allem das Fokussieren
auf den direkten Unterschied beider Mannschaften sowie die Random Parameter Search
des Modells. Final konnte eine 52\% Accuracy erreicht werden.

Wie in der Grafik in Abschnitt drei von Kapitel drei gezeigt, ist es möglich mithilfe
des Modells profitabel zu wetten. Vergleicht man die Profit-Kurve des Modells
mit der der Wettanbieter fällt auf, dass diese im Verlauf der Saison mit zunehmender Sicherheit wesentlich weniger Wetten eingehen. 
Je nach Höhe des Schwellwerts können auch die Wettanbieter mehr Gewinn machen. Beide Modelle sind sich insgesamt ziemlich ähnlich im Bezug auf den generierten Gewinn.
Um also langfristig die Wettanbieter gewinnbringend zu schlagen müsste das Modell
weiter verbessert werden um eine höhere Accuracy zu erzielen. Ein möglicher Ansatz hierzu ist das Hinzufügen weiterer Features. Beispielsweise kann man den Marktwert der Startaufstellung betrachten oder Verletzungen von Spielern mit hinzuziehen. Außerdem erscheint eine Ausweitung der Datenbasis auf weitere Wettbewerbe sinnvoll.

Zusammenfassend lassen sich die Ergebnisse des Projekts folgendermaßen darstellen:
Durch die 52\% Accuracy sowie das profitable Wetten konnte gezeigt werden, dass sich Fußballergebnisse zumindest in
Ansätzen algorithmisch mithilfe von Modellen beschreiben lassen. Damit ist der Use
Case zum Großteil erfüllt. Es ist jedoch kritisch anzumerken, dass auch sehr viele
nicht genau spezifizierbare Zufallsvariablen einen Einfluss auf den Ausgang
eines Spiels haben, sodass sich nie mit 100\% Genauigkeit der Ausgang
eines Fußballspiels vorhersagen lässt.