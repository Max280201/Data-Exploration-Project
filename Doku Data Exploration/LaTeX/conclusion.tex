% !TEX root =  master.tex
\chapter{Fazit}

Abschließend lässt sich festellen, dass sich die Accuracy des Neural Networks
im Verlauf des Projekts Schritt für Schritt durch das optimieren der Datenbasis
und des Modells selber erhöhen ließ. Die größte Erhöhung dabei kam durch das fokussieren
auf den direkten Unterschied der Werte der Mannschaften sowie die Random Paramter Search
des Modells. Final konnte eine 53\% Accuracy erreicht werden.

Wie in der Grafik in Abschnitt drei von Kapitel 3 gezeigt war es möglich mit Hilfe
des Modells profitable zu wetten. Vergleicht man die Profit Kurve es Modells jedoch
mit der der Wettanbieter fällt auf, dass diese im Verlauf der Saison Intervalle mit
höheren Gewinnen aufweisen, ihre Predictions also besser funktionieren. Das lässt sich
darauf zurück führen, dass das Modell mit maximal 65\% Sicherheit das Ergebnis eines Spiels
vorhersagt. Die Modelle der Wettanbieter weißen eine durchschnittlich höhere Sicherheit in Bezug auf die
Ergebnisse der Spiele aus, was dazu führt, dass diese in der Simulation höhere Beträge
einsetzten können und insgesamt auch mehr wetten eingehen als unser Modell.
Um also langfristig die Wettanbieter gewinnbringend zu schlagen müsste das Modell
weiterverbessert werden um eine höhere Accuracy zu erhalten. Mögliche Ansätze hierfür
sind:

Zusammenfassend lassen sich die Ergebnisse des Projekts folgendermaßen darstellen:
Das Projekt war erfolgreich. Durch die 53 prozentige Accuracy sowie das halbwegs
profitbale Wetten konnte gezeigt werden, dass sich Fußballergebnisse zumindest in
Ansätzen algorithmisch mit Hilfe von Modellen beschreiben lassen. Damit ist der Use
Case zum Großteil erfüllt. Es ist jedoch kritisch anzumerken, dass auch sehr viele
nicht genau spezifizierbare Zufallsvariablen einen Einfluss auf den Ausgang
eines Spiels haben, sodass sich nie mit 100\%er Wahrscheinlichkeit der genaue Ausgang
eines Fußballspiels vorhersagen lässt.