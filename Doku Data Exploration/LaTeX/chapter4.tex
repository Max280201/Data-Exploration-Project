\chapter{Ergebnisse}

\section{Abwägung der Algorithmen}

Wie bereits zu sehen war, hat der TSD vergleichsweise schlechte Ergebnisse hervorgebracht. Er ist nicht wirklich geeignet für eine solche Art von Zeitreihe, da er nicht die richtigen Muster in den Daten erkennt. Er erkennt zwar ein Muster, aber nicht ein solches, welches für diese Klassifkation von Nöten wäre. Vielmehr erkennt der TSD Hoch-/Tiefpunkte der Zeitreihe. Nach Testen dieser Herangehensweise fällt aber deutlich auf, dass man damit nicht das gewollte Ergebnis erzielt. 

Der Random Forest dagegen hat vielversprechende Ergebnisse. Diese sind mit ca. 70\% nicht perfekt, aber sie zeigen, dass eine Musterkennung der SQL-Statements möglich ist und funktioniert. Jedoch sobald man das Modell auf Daten nutzt, die nicht im Trainings-/Testdatensatz sind, also völlig ungesehene SQL-Statements, kommt das Modell an seine Grenzen. Es verwechselt sehr oft stündliche und tägliche Daten und klassifiziert wöchentliche Statements ziemlich häufig als tägliche. Allgemein klassifiziert der Algorithmus vergleichsweise viele Statements als täglich. Außerdem kann der Algorithmus nicht die Klassen stündlich (Arbeitszeiten) und täglich (Arbeitstage) erkennen. Das liegt an der geringen Anzahl dieser Muster im Trainings-/Testdatensatz.

\section{Einordnung der Ergebnisse}

Der TSD hat nicht die gewünschten Resultate geliefert. Bei näherer Betrachtung des Random Forest, stößt man auch auf einige Probleme. Sieht man sich die 30\% des Testdatensatzes an, bei denen der Algorithmus falsch klassifiziert hat, fällt auf, dass er wie bereits erwähnt häufig stündliche und tägliche Statements verwechselt und wöchentliche Statements meist als täglich einordnet. Dies lässt sich anhand des Ungleichgewichts im Trainingsdatensatz erklären. Ungefähr 45\% dieser Daten sind Statements mit einem täglichen Muster. Somit ist es nicht verwunderlich, dass der Random Forest so oft Statements als täglich klassifiziert, die in Wahrheit überhaupt kein tägliches Statement sind. Außerdem kann der Algorithmus keine SQL-Statements erkennen, die überhaupt kein Muster haben. Dies ist auch geschuldet am Trainingsdatensatz, da dort Statements ohne Muster nicht berücksichtigt werden.

\section{Fazit und Ausblick}

Als Fazit kann man sagen, dass der Random Forest vielversprechend aussieht. Man kann seine Genauigkeit noch um einiges erhöhen, indem man weitere SQl-Statements in den Trainings-/Testdatensatz aufnimmt. Vor allem sollte man für ein Gleichgewicht innerhalb dieser Daten sorgen. Die herausstehende Präsens von täglichen Statements im Datensatz sollte verringert werden und um Statements mit wöchentlichen, stündlichen (Arbeitszeiten), täglichen (Arbeitstage) und Statements mit überhaupt keinem Muster ergänzt werden. Problematisch ist zusätzlich, dass sich keine SQL-Statements mit quartalsweisen oder jährlichen Mustern finden. Dies ist geschuldet an der Tabelle aus denen die Daten stammen. Diese speichert erst seit Mitte Mai die Daten über SQL-Statements ab. Somit gibt es maximal Statements mit Daten von einem halben Jahr. Bei der intensiven Suche nach monatlichen Statements konnten bisher keine gefunden werden. Abschließend kann man sagen, dass der Algorithmus zeigt, was mit einer Mustererkennung möglich ist, aber es fehlt noch an einem optimierten Datensatz um die Genauigkeit des Modells zu maximieren.